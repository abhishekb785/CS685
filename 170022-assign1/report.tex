\documentclass[11pt]{article}
\usepackage[utf8]{inputenc}
\usepackage{geometry}
 \geometry{
 a4paper,
 total={170mm,257mm},
 left=20mm,
 top=20mm,
 bottom=20mm,
}

\title{CS685A: Assignment 1\\Report: Covid-19 Outburst In India}
\author{\textbf{Question 10}\\
Abhishek Bhatia (170022)}
\date{Odd Semester, 2029-21}

\begin{document}

\maketitle

\section{Overview}

\noindent The first case of corona-virus in India was reported on 30th January, 2020. Since then, there has been an exponential increase in the number of cases talking the total count past 50 lakhs in about 6 months. India has the second largest number of confirmed cases after United States. So, it becomes important to analyze what are the factors responsible for this and how are the cases distributed in the country. It is important to analyze the Hot-Spots(and the cold-spots) at the same time to take appropriate measures in the affected areas. Since, it has become almost impossible to put a nation-wide lock-down again(due to the falling GDP), it becomes all more important to analyze the district-wise picture to take steps accordingly.

\section{Metric Used for Comparison: Z score}
The metric which we are using for comparison between various districts is the Z score. According to wikipedia, Z score is the number of standard deviations by which the value of a raw score is above or below the mean value of what is being observed or measured. In simple terms, the district having the highest z-score among its neighbouring districts(or in the state) can be called as the hotspot for that particular region. Similarly, the district having the minimum Z-score can be called as a coldspot and can be considered to be comparitively safer than the nearby districts.

\section{Hotspot Analysis}
\textbf{Bengaluru Urban} has been the overall most notable hot-spot in the country with a Z-score of around 80. There has been a massive increase in the number of cases in the district, especially in the past two months. Apart from these, the top 5 all time hotspots of the country includes \textbf{Ahmedabad},
\textbf{Lucknow}, \textbf{Surat} and \textbf{Bhopal} with each one of these having z score past 25. One, thing that can be noted here is that the most of the hotspots are either the capital or one of the biggest district of the State. Maharashtra has been a hotspot-state itself right from the beginning with total cases past 12 lakh till now, followed by Andhra Pradesh and Tamil Nadu. Districts of Maharashtra such as Pune, Mumbai, Nagpur as well districts of Gujarat are among the highest number of cases and the cases in these are growing rapidly. \\
Apart from looking at this overall scenario, it is also important to look at the recent hotspots. It is a matter of concern that Districts such as West Tripura, Puducherry,Raipur are among the top recent hot-spots of their states, though they had relatively lower number of cases earlier. This shows that the cases has increased rapidly in these areas. Disctricts such as Mysuru, Lucknow and Surat continue to remain on the top among its neighbouring districts even in the recent weeks


\section{Coldspot Analysis}
Similar to the hotspot analysis, we can also do a cold spot analysis as to figure out which district is lesser affected as compared to it's nearby districts(or in the state). It is important to look at these cold spot analysis to figure out what are the measure taking by these districts to remain relatively safer. Hard to reach areas of the country such as Kinnaur, Unokoti, South Tripura were among the most prominent coldspots. Smaller districts such as Karur, Mahe are among the recent cold spots since the rate of increase of cases has been relatively low.

\section{Conclusion}
As it is evident from the above analysis that bigger and more popular cities are worst hit by this pandemic. Ther has been rapid increase in cases, especially in Maharashtra, Andhra Pradesh and some districts of Gujarat. Though, some of the districts which were earlier hotspots have recovered, possibly through their protective measures showing that there is still some hope and the rate can decreased in the future.

\end{document}
